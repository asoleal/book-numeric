En ésta sección se describe el uso del cliente docker dentro de gitpod
para tener un sistema funcional, flexible y adaptable que incorpore las herramientas de DUNE
online. Es un punto de inicio o de base para el desarrollo de nuevas propuestas.

Utilizar una imagen preisntalada con dunepython, jupyternotebook y su forma de utilización.

Para mayor referencia se utiliza la página oficial de docker, en donde se explica de manera 
completa la explicación paso a paso de cómo se instala, guías y manuales, etc.

La clave de la sección es tratar de explicar la funcionalidad de docker en el manejo científico,
en la página está explicada de manera general, la idea es explicarlo de manera particular.