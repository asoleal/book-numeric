Presentar una serie de ejemplos y ejercicios clásicos, con sus descripciones
caracaterísticas, condiciones iniciales o de frontera.  Una selección 
de teoría, partiendo por ejemplo de las leyes de conservación y de las 
leyes empíricas, Ley de Fourier, Ley de Darcy, Leyes de Maxwell, Leyes 
en el tránsito, flujos en suelos y plantas, circuitos.
Ecuación de Poisson, Ecuación de Onda, etc.

Es importante la explicación de la condiciones de frontera tipo Dirichlet y Neuman.
Posiblemente, en los libros de Heildeberg. También revisar el libro del profesor 
Hernán Estrada.