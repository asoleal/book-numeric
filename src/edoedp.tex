Presentar una serie de ejemplos y ejercicios clásicos, con sus descripciones
caracaterísticas, condiciones iniciales o de frontera.  Una selección 
de teorías, partiendo por ejemplo de las leyes de conservación y de las 
leyes empíricas, Ley de Fourier, Ley de Darcy, Leyes de Maxwell, Leyes 
en el tránsito, flujos en suelos y plantas, circuitos.
Ecuación de Poisson, Ecuación de Onda, etc.

Es importante la explicación de la condiciones de frontera tipo Dirichlet y Neuman.
Posiblemente, en los libros de Heildeberg. También revisar el libro del profesor 
Hernán Estrada.

\section{La derivada}
El modelamiento matemático es una técnica que utilizan principalmente los matemáticos e ingenieros 
para comprender, simular y predecir el comportamiento de sistemas físicos.  Newton fue uno de 
los precursores, buscaba predecir el comportamiento de los cuerpos que se movían, por ejemplo
tratar de explicar la rotación de los planetas alrededor del sol, o un coche que se moviera en una 
dirección particular.  Para tratar de comprender lo que ocurría, inventó algunos conceptos que son 
muy utilizados hoy en día, por ejemplo el concepto de fuerza, y a diferencia de lo que se había planteado
anteriormente, Newton estableció que todo cuerpo tiende a mantener su estado, excepto que haya una
fuerza externa que cambié su estado inicial.  

Estableció entonces tres leyes que rigen los movimientos de los cuerpos, a saber:

\section{Leyes de conservación}

\section{Casos especiales de las leyes de conservación}

\section{Ejemplos de modelos matemáticos}


