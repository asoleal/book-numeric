La idea general del libro es que una persona que tenga los conocimientos básicos
de ecuaciones, algebra y programación, pueda llegar al libro y tener una 
guía que le sirva para resolver problemas con DUNE.

El objetivo del libro es mostrar la solución de problemas especiales utilizando DUNE, 
el método de líneas como por ejemplo \url{https://www.youtube.com/watch?v=4R-BXbL183E},
Problemas resueltos por el profesor Hernan Estrada en su libro física computacional ejemplo,
el problema de ecuaciones diferenciales no lineales como la del memristor, que los 
Software dificilmente resuelven.

Resolver modelos como los que propone Logan, o algunos problemas del Agro, o de otras 
ramas de la ciencia, no lineales, especiales.

\section{Historia de DUNE}
\section{Filosofía del programa}
\section{Estructura de DUNE}
\section{DUMUX, Y OTROS SOFTWARE BASADOS EN DUNE }
\section{Aplicaciones}
\subsection{Ecuación de Poisson}
\subsection{Otras aplicaciones}
\section{Dune FEM y PDElab}


En la ecuación \eqref{eq:integral01}
\begin{equation}\label{eq:integral01}
 \int_a^bf(x)dx
\end{equation}
En la figura \ref{fig:prueba01}
\begin{figure}\label{fig:prueba01}
    \includegraphics[scale = 0.5]{example-image}
\end{figure}
En la tabla \ref{tabla:matriz01}
\begin{table}[H]\label{tabla:matriz01}
    \begin{tabular}{lllll}
     2& 3 & 4 & 5 &  \\
     -1& 0 & 2 & 5 &  
    \end{tabular}
\end{table}
Las gráficas:

We are working on
\begin{tikzpicture}
\draw (-1.5,0) -- (1.5,0);
\draw (0,-1.5) -- (0,1.5);
\end{tikzpicture}.
